\documentclass[letterpaper,10pt]{article}

\usepackage{titling}
\usepackage{listings}
\usepackage{url}
\usepackage{setspace}
\usepackage{subfig}
\usepackage{sectsty}
\usepackage{pdfpages}
\usepackage{colortbl}
\usepackage{multirow}
\usepackage{relsize}
\usepackage{amsmath}
\usepackage[compact]{titlesec}
\usepackage[default]{droidserif}
\usepackage[T1]{fontenc}
\usepackage{tikz}
\usetikzlibrary{arrows,automata,shapes,matrix,chains,scopes,positioning,calc}
\tikzstyle{block} = [rectangle, draw, fill=blue!20, 
    text width=2.5em, text centered, rounded corners, minimum height=2em]
\tikzstyle{bw} = [rectangle, draw, fill=blue!20, 
    text width=4em, text centered, rounded corners, minimum height=2em]

\definecolor{namerow}{cmyk}{.40,.40,.40,.40}
\definecolor{namecol}{cmyk}{.40,.40,.40,.40}

\let\LaTeXtitle\title
\renewcommand{\title}[1]{\LaTeXtitle{\textsf{#1}}}


\addtolength{\oddsidemargin}{-1.000in}
\addtolength{\evensidemargin}{-0.500in}
\addtolength{\textwidth}{2.0in}
\addtolength{\topmargin}{-1.000in}
\addtolength{\textheight}{1.75in}
\addtolength{\parskip}{\baselineskip}
\setlength{\parindent}{0in}
\renewcommand{\baselinestretch}{1.5}

\singlespace

\date{}
\title{\bf\LARGE ECE~155: Engineering Design with Embedded Systems (Spring~2016) \\ - \\ Section LEC002}
\author{Department of Electrical \& Computer Engineering \\
                University of Waterloo}


\begin{document}

\maketitle
\vspace{-0.5in}

\section*{About the Course}

This course focuses on three areas: software development, embedded systems, and project management. Programming for embedded systems means we will learn about event-driven programming, debugging, and timers. We also discuss project management, such as planning, specifications, modelling, and scheduling. Generally, this class contains a lot of things that students of the old curriculum found were taught very late in the program when they needed to be up front. This class has a lab component as well. In the labs we will be developing applications for Android devices which use the various sensors inside the device.

\paragraph{Undergraduate Calendar Description} ``Introduction to embedded systems, review of engineering design and analysis principles, software development life cycle, integrated development environments, use of software requirements and specifications, unified modelling language and documentation, event handling, simulation, project management, project scheduling, testing, verification, and maintenance considerations.''

ECE~150, Fundamentals of Programming, is a prerequisite for this course. As you might expect, this class builds upon the programming knowledge you gained in that course. This course uses Java, which is an object oriented programming (OOP) language. We will spend the first week introducing the fundamentals of OOP to help in the transition from more procedural languages like C/C++. If you had a hard time in ECE~150 or you still feel uncomfortable with programming in general, I suggest using some time at the start of the term (when you are comparatively less busy) to address this. Regardless, I suggest that everyone spend a little time to get used to Java so that you can complete the labs.

\paragraph{Course Objectives} At the end of this course, students should be able to:
\begin{enumerate}
\itemsep0em
        \item describe the major blocks comprising common embedded systems;
        \item program an event-driven embedded system (Android);
        \item compare and contrast different software development life cycles and situations in which they work well;
        \item describe the engineering design process, planning and estimation, reviews, simulation, and software maintenance;
        \item collect and document requirements for, and design, a simple software system;
        \item use modern software development tools and concepts, including: integrated development environments, version control, UML, refactoring, and unit tests.
\end{enumerate}

\paragraph{Course Website} As is standard, information will be posted on LEARN.
For additional information about First Year Engineering, check the First Year Office's website:
\texttt{https://uwaterloo.ca/engineering/\\current-undergraduate-students/first-year}


\paragraph{Textbook} There is no required textbook for this class.



\paragraph{Class Schedule}

A brief summary of the schedule for this class follows. Your online schedule of classes will tell you which tutorial and lab sections you are enrolled in. I've done my best to copy this from the timetable provided by the department. Please double-check your timetable to make sure you know which section you are in, and when/where to attend your labs and tutorials.

Note that there are two lecture sections of ECE 155 in Spring 2016. We are in section LEC002. Both sections are being run as a single course, and will cover the same material. TAs, tutorials, labs, and exams are the same for both sections. However, individual lectures will naturally have some differences in terms of delivery. Slides (PDF format) specific to our lecture section will be posted on LEARN.

\begin{table}[h]
        \begin{center}
        \begin{tabular}{l l l l}
				Regular Lectures & LEC002 & Tu 08:30-09:50 & RCH 302 	\\
								 & 		  & Th 08:30-09:50 & RCH 302 	\\
				\hline
				Make-up Lectures*& LEC002 & Fr 11:30-12:20 & RCH 302 	\\
				\hline			
				Tutorials 		 & TUT104 & Th 14:30-15:20 & DWE 3522A 	\\
						  		 & TUT105 & Tu 14:30-15:20 & RCH 309   	\\
						  		 & TUT106 & We 14:30-15:20 & RCH 204   	\\
				\hline			
				Labs** 	  		 & LAB204 & We 13:30-16:20 & E2-3344	\\
								 & LAB205 & Th 13:30-16:20 & E2-3344	\\
								 & LAB206 & Tu 13:30-16:20 & E2-3344	\\
				\hline
			\end{tabular}
        \end{center}
        Midterm: Friday June 17, 8:30am-10:20am				\\
		*    Make-up Lecture Dates: TBD						\\
		**   Labs are held on weeks 1, 3, 5, 7, 9, and 11 	\\
\end{table}

\textit{Midterm Week}: June 13-17. Classes are not held during the midterm week.

\textit{Schedule Oddity}: Tuesday, July 26 (last day of lectures) follows the \textit{Friday} schedule of classes.

\textit{Make-Up Lecture Dates}: May 13, May 27, June 10, June 24, July 8, July 22. \\
We must use three of these dates to fit in all of the required lecture hours. The specific dates that we will use are TBD.

\textit{Lab Dates}:

\textit{LAB204}: May 4, May 18, June 1, June 15, June 29, July 13 \\
\textit{LAB205}: May 5, May 19, June 2, June 16, June 30, July 14 \\
\textit{LAB206}: May 3, May 17, May 31, June 14, June 28, July 12 

\textit{Lab Submission Dates}:

Please note the following submission due dates for the labs (all times are Eastern Daylight Savings Time): \\

\begin{table}[h]
        \begin{center}
        \begin{tabular}{l|l|l|l|l}
        	\textbf{Section} & \textbf{Lab 1} & \textbf{Lab 2} & \textbf{Lab 3} & \textbf{Lab 4} \\
			\hline
			204 & ??? 23:59 & ??? 23:59 & ??? 23:59 & ??? 10:00 \\
			\hline
			205 & ??? 23:59 & ??? 23:59 & ??? 23:59 & ??? 10:00 \\
			\hline
			206 & ??? 23:59 & ??? 23:59 & ??? 23:59 & ??? 10:00 \\
        \end{tabular}
        \end{center}
\end{table}

\textit{Final Exam}: The final exam period runs from August 2 to August 13. The registrar's office will announce the exact date, time, and location of the final exam. Our exam could be on any date in this period. Make sure that any travel plans you have are AFTER the last day of exams. Student travel plans are \emph{not} considered an acceptable reason for missing an exam. When the final exam date is announced, please alert your instructor immediately if you have a conflict.

\section*{Course Staff}

For all course staff, office hours are by appointment. Please keep in mind that course staff have other responsibilities, so it may not be possible for them to meet with you at the last minute. Similarly, please do not expect that course staff will answer an e-mail sent seven minutes before the final exam.

\begin{table}[h]
        \begin{center}
        \begin{tabular}{l|l|l|l}
        Instructor & Scott Chen & EIT-4006 & \texttt{w25chen@uwaterloo.ca} \\
        \hline
        Instructor & Michael Giannikouris & EIT-4007 & \texttt{mgiannik@uwaterloo.ca} \\
        \hline
        Lab Instructor & Sanjay Singh & DC-2629 & \texttt{ssingh@uwaterloo.ca}\\
        \hline
        TA & Hossameldin Amer &  & \texttt{h2amer@uwaterloo.ca} \\
        TA & Liang Dong &  & \texttt{l28dong@uwaterloo.ca} \\
        TA & Mahmoud Khalafalla &  & \texttt{mkhalafa@uwaterloo.ca} \\
        TA & Reinier Torres & & \texttt{rtorresl@uwaterloo.ca} \\
        TA & Xueren Wang & & \texttt{x537wang@uwaterloo.ca} \\
        \end{tabular}
        \end{center}
\end{table}


\paragraph{About Mr. Giannikouris}
I (perhaps surprisingly) have a Bachelor and Masters degree in Mechanical Engineering from Waterloo. I got into software development during my graduate studies, when I was heavily involved in the University of Waterloo Alternative Fuels Team. If you have any interest in getting some hands on engineering experience I highly recommend checking out the teams in E5. I have also been a TA for a few different undergraduate courses, including the embedded operating systems course in Mechatronics Engineering (MTE 241).

I am a sessional lecturer, which means that teaching is a part time job. Additionally, I have a full-time job as an embedded software developer in Waterloo. Therefore, I'm not likely to be found in my office during the day. If you want to meet, please send me an e-mail and we will set up an appointment.

\paragraph{About the Lab Instructor.}

The lab instructor is responsible for the labs in this course, and can answer your all your lab-related questions. You may contact the lab instructor to arrange some more time with the lab equipment, if you need. The lab instructor will be present at the labs and will supervise them.

\paragraph{About the Teaching Assistants.}

Teaching assistants can help you with the course material, including tutorials, labs, and exams. They will be present in the labs and will conduct the tutorials.

\section*{Grading Scheme}

Your class grade is comprised of your marks in the labs $l_{i}$, the midterm exam $m$, and the final exam $f$.

The lab grade $L$ is calculated as: $\sum_{i=1}^{4}l_{i}$.

The weight of the midterm $W_{m}$ is $25$. The weight of the labs $W_{l}$ depends on your final exam grade $f$ and follows this formula:
\[
 W_{l} = \begin{cases}
        25      & $if~$ f \ge 60\\
        f - 35  & $if~$ 40 \le f < 60\\
        5       & $if~$ f < 40
        \end{cases}
\]

The weight of the final is $W_{f}$ is $100 - (W_{l} + W_{m})$.

Your final grade is calculated as: $\frac{W_{l}}{100}L + \frac{W_{m}}{100}m + \frac{W_{f}}{100}f $.

If you miss the midterm exam and provide a valid reason, the grading scheme will change such that the final exam is worth 75\%. If you miss the midterm exam without a valid reason then you will get a grade of 0 for the midterm exam and the grading scheme above still applies. A valid reason means either a verification of illness form or extenuating circumstances that are handled on a case-by-case basis. It is in your best interest to make your instructor aware of the reason for any missed deliverable as soon as possible. More information can be found in the Attendance and Illness section of the syllabus.

The University rules say if you miss the final exam, without an acceptable reason, your grade in the class will be DNW - Did Not Write. This is very undesirable. Show up for the final exam.

See also the section about late submissions for labs under the section Course Policies.

I cannot tell you about your final exam grade or your total course grade until after marks become visible in Quest. Please do not e-mail your instructor (or any of the TAs, or the Lab Instructor) after the final exam asking about your grades. You will simply have to be patient.


\section*{Labs}
This class includes Labs, which you will complete in groups of three. Unlike the initial offerings of this class, we will use Android smartphones in the labs instead of Lego robots. Lego robots sound pretty cool, but they were difficult to work with and not available outside the lab hours (and students hated that). There will be UW-owned Android tablets (Nexus 7!) available in the labs, and there are UW-owned Nexus One phones on reserve for 3-day loan at the Davis Centre Library. That said, Android phones and tablets are very popular; you can use your own in the labs if you choose.

You have six lab sessions of three hours each. During the lab sessions, you can work on the labs, the tablets will be in the lab room, and the TAs will be available to help you (this is how upper-year labs work). You will also demonstrate your previous work to the TAs (if applicable).

TAs will mark your labs based on the quality and performance of your design. You must submit your source code to the course Subversion (\texttt{svn}) repository and demonstrate your software to the TAs during the lab demo period. Each lab (other than Lab 0; Lab 0 is an introductory session and is not graded) will be graded according to the lab demonstration rubric. Each lab will have its own rubric, released in advance of the lab. We will also provide feedback on your implementation.

After each lab session, you may continue to work on your solution until the due date, at which time you must have committed your code to the repository. In the lab session immediately following the due date for the lab, you must be prepared to demonstrate your work. The exception to this is Lab 4, where you must present your work at the end of the final lab session.  If your group is not in the lab, the source code is not submitted into \texttt{svn}, or you are not ready to present your work when it is your turn, your group will receive a 0 grade for the lab being graded.

Please note the due dates for the labs in the Class Schedule section of this syllabus. On grounds of fairness, late lab submissions will not be accepted. The timestamp of the ECE subversion server will be considered the official time of submission of any deliverable.

Your lab code will be checked for plagiarism using Moss (Measure Of Software Similarity). You may request to opt out of the automatic screening by sending a formal written letter to your instructor explaining why; a meeting will then follow to discuss the subject with the instructor.

\section*{Course Policies}

\paragraph{Collaboration \& Plagiarism}

Plagiarism (taking credit for work that others did) is not permitted. This applies to source code as well as exams. The course staff will be checking for plagiarism using a variety of different methods. Any cases of plagiarism detected by course staff will be reported according to University policy (see the University Policy section of the syllabus).

It is expected you will collaborate with your lab partners. Your lab submissions are joint efforts and the work you submit in is that of your lab group. Note, however, that grades are still individual and if one group member is not contributing to the lab work at all, that student may receive a zero in the lab.

You may discuss ideas and design alternatives with other groups, and help other groups debug small fragments of code. However, each group must submit their own, independently-developed code for each lab. I suggest that you avoid looking at the code of other groups entirely, but if you do, you should not look at any part of their code that you will be writing for your own group.

Groups are \textbf{not} permitted to share code electronically or in written form, unless such sharing has been clearly documented and acknowledged in the submitted work. An acknowledged fragment of code will not be eligible for grading as part of the lab, but it will also not result in disciplinary penalties. Acknowledgements must include the name of the providing group and the date of the collaboration. It is very important to acknowledge the work of others if you use it.

For the record, all members of a group take responsibility for a submitted piece of work. Thus, if a member of your lab group copies some code without acknowledgement and it is submitted under all of your names, even if you did not know that the code was copied, you will all be held responsible. Put another way, if you copy some code without acknowledgement, all of your lab partner(s) will jointly be held responsible for your actions. 

Please don't plagiarize, it creates a very bad situation for everyone.

Responsibility for avoiding plagiarism also rests with the person whose work may be copied. It's very important that you do not make your lab code available to others. For example, do not publish your code to any kind of public forum or repository where others could access it without your authorization. GitHub is a prime example of somewhere you do \textbf{not} want to store your lab code. Making your lab code available to others can expose you to a possible academic offence in the event that someone copies the code for their own submission, even if you did not give permission.

I want to emphasize that the course staff take the issue of plagiarism very seriously, and so does the University of Waterloo. If you are uncertain about this subject, please seek some guidance. There are many resources available to you. You can check the university policies, talk to the course instructor or lab instructor, visit the First Year Engineering Office, et cetera.

Or, let's sum this up in three short instructions:
\begin{enumerate}
	\item Always acknowledge the work of others. 
	\item Do not make your work available for others to copy.
	\item If you are uncertain, ask!
\end{enumerate}

\paragraph{Late Submissions} Late lab submissions will not be accepted; see the section on labs.

The timestamp of the ECE subversion server will be considered the official time of submission of any lab. Don't wait until a few minutes before the deadline to try to submit your code. Things sometimes happen at inconvenient times (internet outage, power outage, any kind of outage really).

\paragraph{Re-marking}

If you believe that your grade on an a written, submitted deliverable (e.g., a midterm exam question) is incorrect or unfair, you may ask that it be re-marked. To request that a question be re-marked, you will need to submit your request on a sheet of paper, in writing, to your course instructor. You may submit it in person, or ask an administrative assistant at the Electrical \& Computer Engineering undergraduate office to put it in your instructor's mailbox. Please do not submit your request to a TA or the lab staff. 

When you submit your request, it should include the following: (1) Your name and student ID number; (2) a clear indication of which question or part of the deliverable is to be re-marked; and (3) an explanation of why you believe the grade assigned was incorrect.

If you received the marked version of the deliverable back (e.g., the midterm exam), please submit that alongside your written request. Staple them (not paperclip, not some sort of origami) together so they do not get separated.

Items for re-marking will be accepted any time before the final exam. Be forewarned, when a deliverable is being re-marked, your grade could go up, it could stay the same, or it could go down. Regardless of the result, this will be the new grade. You will be notified of the outcome and an attempt will be made to return the deliverable you submitted (if any).

Please note that lab demonstrations cannot be re-marked.

\paragraph{Extra Credit}
In this class, there will be no opportunities to earn extra credit. Make-up assignments, labs, or examinations will not be offered under any circumstances.

\paragraph{Attendance \& Illness}

Attending lectures is absolutely in your best interest. It is also a very efficient use of your time, if you pay attention and participate in discussions. Attending lectures ensures that you receive all of the course material. We will be using the whiteboard, and not everything we say or do in lectures is going to be in the course slides. It's also very important to see how the instructor is using the course material in examples, and what relative importance is being placed on each topic. This is rather helpful when preparing for exams. Finally, attendance gives you an opportunity to ask questions, and perhaps more importantly, to hear the answers to questions posed by your classmates (questions that you didn't even know to ask).

That said, this is University and ultimately the decision is yours. Attendance in lectures is not taken and not graded. Attendance is not taken in tutorials or labs, but at least one member of your lab group must to be present in the labs to present lab deliverables.

During the term, you may need arrive/depart while a lecture is in progress because of co-op interviews. This is not a problem, as long as you are not disruptive when arriving/departing.

If you feel ill, you should seek appropriate medical attention. If you miss an exam for health reasons, you need a verification of illness form. Forms can be completed by the physicians at Health Services. Your completed verification of illness form should be presented to the First Year Engineering Office for verification, \textbf{not} directly to the course staff. If you anticipate missing a deliverable deadline or an examination for a non-medical reason, you should contact your instructor as soon as you are aware of the problem. Given sufficient notice, alternate arrangements may be possible. Alternate arrangements are rare and at the discretion of the course instructor.


\section*{University Policies}

\paragraph{Academic Integrity}
In order to maintain a culture of academic integrity, members of the University of Waterloo community are expected to promote honesty, trust, fairness, respect and responsibility. Check \texttt{www.uwaterloo.ca/academicintegrity/} for more information.

\paragraph{Grievance}
A student who believes that a decision affecting some aspect of his/her university life has been unfair or unreasonable may have grounds for initiating a grievance. Read Policy 70, Student Petitions and Grievances, Section 4, \texttt{adm.uwaterloo.ca/infosec/Policies/policy70.htm} \\
If in doubt, contact the department's administrative assistant, who will provide further assistance.

\paragraph{Discipline}
A student is expected to know what constitutes academic integrity (see above section) to avoid committing an academic offence, and to take responsibility for his/her actions. A student who is unsure whether an action constitutes an offence, or who needs help in learning how to avoid offences (e.g., plagiarism, cheating) or about ''rules'' for group work/collaboration should seek guidance from the course instructor, academic advisor, or the undergraduate Associate Dean. For information on categories of offences and types of penalties, students should refer to Policy 71, Student Discipline, \texttt{www.adm.uwaterloo.ca/infosec/Policies/policy71.htm} . For typical penalties check Guidelines for the Assessment of Penalties, see \\\texttt{www.adm.uwaterloo.ca/infosec/guidelines/penaltyguidelines.htm} .

\paragraph{Appeals}
A decision made or penalty imposed under Policy 70 (Student Petitions and Grievances) (other than a petition) or Policy 71 (Student Discipline) may be appealed if there is a ground. A student who believes he/she has a ground for an appeal should refer to Policy 72 (Student Appeals)\\~\texttt{www.adm.uwaterloo.ca/infosec/Policies/policy72.htm}.

\paragraph{Privacy}
Questions about the collection, use, and disclosure of personal information by the University, should be directed to the Freedom of Information and Privacy Coordinator, Secretariat, University of Waterloo, 200 University Avenue West, Waterloo, Ontario, Canada N2L 3G1. The email address of the Freedom of Information and Privacy Coordinator is \texttt{fippa@uwaterloo.ca}. See also University of Waterloo Policy 19: Access to and Release of Student Information; Information and Privacy.
\\ \texttt{https://uwaterloo.ca/secretariat/policies-procedures-guidelines/policy-19}

\paragraph{Note for Students with Special Needs}
The AccessAbility Services (formerly known as OPD) located in Needles Hall, Room 1132, collaborates with all academic departments to arrange appropriate accommodations for students with disabilities without compromising the academic integrity of the curriculum. If you require academic accommodations to lessen the impact of your disability, please register with the AccessAbility Services office at the beginning of each academic term.

\end{document}