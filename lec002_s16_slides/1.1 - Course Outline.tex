\documentclass{beamer}
\usetheme{metropolis}           % Use metropolis theme
\metroset{numbering=fraction}
\usepackage{tikz}
\usetikzlibrary{arrows,positioning,shapes.geometric}
\usepackage{float}
\usepackage{makecell}
\usepackage{fancyvrb}
\usepackage[export]{adjustbox}
\usepackage{caption}
\title{Lecture 1.1 \\ ECE 155 Course Outline}
\date{\today}
\author{Patrick Lam \\ Jeff Zarnett \\ Michael Giannikouris}
\institute{Department of Electrical and Computer Engineering}
\setbeamertemplate{caption}{\raggedright\insertcaption\par}
\setbeamersize{text margin left=12pt,text margin right=12pt}
\newcommand{\putat}[3]{\begin{picture}(0,0)(0,0)\put(#1,#2){#3}\end{picture}} % just a shorthand

\begin{document}

  \maketitle
  \section{Course Syllabus}
  
  %Course Objectives
  \begin{frame}{Course Objectives}
    \begin{itemize}
    \begin{small}
      \item describe the major components of common embedded systems
      \item program an event-driven embedded system (Android) \includegraphics[width=1em,height=1em]{img/android_peeking}
      \item compare and contrast different software development life cycles and situations in which they work well
      \item describe the engineering design process, planning and estimation, reviews, simulation, and software maintenance
      \item collect and document requirements for, and design, a simple software system
      \item use modern software development tools and concepts, including integrated development environments (IDEs), version control,  UML, refactoring, and unit tests
	\end{small}
    \end{itemize}
  \end{frame}
  
	%Course Schedule
	\begin{frame}{Course Schedule}
		\begin{center}
			\begin{tabular}{l l l l}
				Regular Lectures & LEC002 & Tu 08:30-09:50 & RCH 302 	\\
								 & 		  & Th 08:30-09:50 & RCH 302 	\\
				\hline
				Make-up Lectures*& LEC002 & Fr 11:30-12:20 & RCH 302 	\\
				\hline			
				Tutorials 		 & TUT104 & Th 14:30-15:20 & DWE 3522A 	\\
						  		 & TUT105 & Tu 14:30-15:20 & RCH 309   	\\
						  		 & TUT106 & We 14:30-15:20 & RCH 204   	\\
				\hline			
				Labs** 	  		 & LAB204 & We 13:30-16:20 & E2-3344	\\
								 & LAB205 & Th 13:30-16:20 & E2-3344	\\
								 & LAB206 & Tu 13:30-16:20 & E2-3344	\\
				\hline
			\end{tabular}
		\end{center}		
		Midterm: Friday June 17, 8:30am-10:20am				\\
		*    Make-up Lecture Dates: TBD						\\
		**   Labs are held on weeks 1, 3, 5, 7, 9, and 11 	\\
	\end{frame}
	
	%Course Staff
	\begin{frame}{Course Staff}
		\begin{center}
		\begin{footnotesize}
		\begin{tabular}{l|l|l|l}
        Instructor & Scott Chen & EIT-4006 & \texttt{w25chen@uwaterloo.ca} \\
        \hline
        Instructor & Michael Giannikouris & EIT-4007 & \texttt{mgiannik@uwaterloo.ca} \\
        \hline
        Lab Instructor & Sanjay Singh & DC-2629 & \texttt{ssingh@uwaterloo.ca}\\
        \hline
        TA & Hossameldin Amer &  & \texttt{h2amer@uwaterloo.ca} \\
        TA & Liang Dong &  & \texttt{l28dong@uwaterloo.ca} \\
        TA & Mahmoud Khalafalla &  & \texttt{mkhalafa@uwaterloo.ca} \\
        TA & Reinier Torres & & \texttt{rtorresl@uwaterloo.ca} \\
        TA & Xueren Wang & & \texttt{x537wang@uwaterloo.ca} \\
        \end{tabular}
        \end{footnotesize}
		\end{center}
	\end{frame}
	
	%Grading Scheme
	\begin{frame}{Course Grading}
	Your class grade is comprised of your marks in the labs $l_{i}$, the midterm exam $m$, and the final exam $f$.

The lab grade $L$ is calculated as: $\sum_{i=1}^{4}l_{i}$.

The weight of the midterm $W_{m}$ is $25$. The weight of the labs $W_{l}$ depends on your final exam grade $f$ and follows this formula:
\[
 W_{l} = \begin{cases}
        25      & $if~$ f \ge 60\\
        f - 35  & $if~$ 40 \le f < 60\\
        5       & $if~$ f < 40
        \end{cases}
\]

The weight of the final is $W_{f}$ is $100 - (W_{l} + W_{m})$.

Your final grade is calculated as: $\frac{W_{l}}{100}L + \frac{W_{m}}{100}m + \frac{W_{f}}{100}f $.
	\end{frame}
	
	%Course Labs
	\begin{frame}{Course Labs}
		\textit{LAB204}: May 4, May 18, June 1, June 15, June 29, July 13 \\
		\textit{LAB205}: May 5, May 19, June 2, June 16, June 30, July 14 \\
		\textit{LAB206}: May 3, May 17, May 31, June 14, June 28, July 12 
	\end{frame}
  
	%Course Policies
	\begin{frame}{Course Policies}
	Please see details in the syllabus for: \\
		\begin{itemize}
			\setlength\itemsep{1em}
			\item Collaboration \& Plagiarism
			\item Late Submissions
			\item Re-marking
			\item Extra Credit
			\item Attendance \& Illness
		\end{itemize}
	\end{frame}
	
	%University Policies
	\begin{frame}{University Policies}
		Some University policies you will want to be familiar with:
		\vspace{1em}
		\begin{itemize}
			\setlength\itemsep{0.75em}
			\item \normalsize{Academic Integrity} \tiny{(www.uwaterloo.ca/academicintegrity/)}
			\item \normalsize{Grievance} \tiny{(adm.uwaterloo.ca/infosec/Policies/policy70.htm)}
			\item \normalsize{Discipline} \tiny{(www.adm.uwaterloo.ca/infosec/Policies/policy71.htm)}
			\item \normalsize{Penalties} \tiny{(www.adm.uwaterloo.ca/infosec/guidelines/penaltyguidelines.htm)}
			\item \normalsize{Appeals} \tiny{(www.adm.uwaterloo.ca/infosec/Policies/policy72.htm)}
			\item \normalsize{Privacy} \tiny{(https://uwaterloo.ca/secretariat/policies-procedures-guidelines/policy-19)}
			\item \normalsize{Special Needs} \tiny{(AccessAbility Services, Needles Hall, Room 1132)}
		\end{itemize}
	\end{frame}  

\end{document}