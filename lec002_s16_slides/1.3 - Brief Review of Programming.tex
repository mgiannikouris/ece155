\documentclass{beamer}
\usetheme{metropolis}           % Use metropolis theme
\metroset{numbering=fraction}
\usepackage{tikz}
\usetikzlibrary{arrows,positioning,shapes.geometric}
\usepackage{float}
\usepackage{makecell}
\usepackage{fancyvrb}
\usepackage[export]{adjustbox}
\usepackage{caption}
\title{Lecture 1.3 \\ Evolution of Programming}
\date{\today}
\author{Patrick Lam \\ Jeff Zarnett \\ Michael Giannikouris}
\institute{Department of Electrical and Computer Engineering}
\setbeamertemplate{caption}{\raggedright\insertcaption\par}
\setbeamersize{text margin left=12pt,text margin right=12pt}
\newcommand{\putat}[3]{\begin{picture}(0,0)(0,0)\put(#1,#2){#3}\end{picture}} % just a shorthand

\begin{document}
\maketitle

\section{Programming Paradigms}
  
\begin{frame}{Programming Paradigms}
	\begin{description}
		\item[Imperative]		
		Programs are a set of instructions that change the state of the program. Think assembly language.
		
		\item[Structured]
		Derived from imperative programming. Organize program into blocks of code. Introduces control structures:
		
		\begin{itemize}
			\item Sequence	(executing blocks of code in order)
			\item Selection (if statements)
			\item Iteration (loops)
		\end{itemize}
		
		\item[Procedural]
		Derived from structured programming. Program is composed of a set of procedure calls (functions) and data structures (variables). We have scope (e.g. local variables in a function).

	\end{description}
\end{frame}  
  
\begin{frame}{Programming Paradigms}

	\begin{description}
		\item[Object-Oriented (OOP)]
		Program is organized into a set of objects. An object represents a single "thing". Objects contains both data and methods that pertain to that thing. Multiple objects interact to execute the program.
		
		\item[Event-Driven]
		Program flow is driven by "events". In an embedded system, events often come from the environment (e.g. sensors) or timers.
	\end{description}
\end{frame}  

\section{Procedural vs. Object-Oriented Programming}

\begin{frame}[fragile]{Procedural vs. Object-Oriented Programming}

To illustrate the differences, calculate the area of a circle both ways.

\centering
\begin{tabular}{@{}m{0.5\textwidth} | m{0.5\textwidth}@{}}
C					&					Java	\\
\hline
\begin{Verbatim}[fontsize=\tiny]
double circle_area(double radius) {
  return 3.14159 * radius * radius;
}

double circle_circum(double radius) {
  return 2 * 3.14159 * radius;
}

int main(void) {
  double radius;
  double area;
  double circumference;

  radius = 1.6;
  area = circle_area(radius);
  circumference = circle_circum(radius);
	
  return 0;
}
\end{Verbatim}	

\end{tabular}

\end{frame}

\begin{frame}[fragile]{Procedural vs. Object-Oriented Programming}

To illustrate the differences, calculate the area of a circle both ways.

\centering
\begin{tabular}{@{}m{0.5\textwidth} | m{0.5\textwidth}@{}}
C					&					Java	\\
\hline
\begin{Verbatim}[fontsize=\tiny]
typedef struct {
  double radius;
  double pi;
} circle_t;

double circle_area(circle_t c) {
  return c.pi * c.radius * c.radius;
}

double circle_circum(circle_t c) {
  return 2 * c.pi * c.radius;
}

int main(void) {
  circle_t circle;
  circle.radius = 1.6;
  circle.pi = 3.14159
  
  double area = circle_area(circle);
  double circumference = circle_circum(circle);
	
  return 0;
}
\end{Verbatim}	

\end{tabular}

\end{frame}

\begin{frame}[fragile]{Procedural vs. Object-Oriented Programming}

To illustrate the differences, calculate the area of a circle both ways.

\centering
\begin{tabular}{@{}m{0.5\textwidth} | m{0.5\textwidth}@{}}
C					&					Java	\\
\hline
\begin{Verbatim}[fontsize=\tiny]
typedef struct {
  double radius;
  double pi;
} circle_t;

double circle_area(circle_t c) {
  return c.pi * c.radius * c.radius;
}

double circle_circum(circle_t c) {
  return 2 * c.pi * c.radius;
}

int main(void) {
  circle_t circle;
  circle.radius = 1.6;
  circle.pi = 3.14159
  
  double area = circle_area(circle);
  double circumference = circle_circum(circle);
	
  return 0;
}
\end{Verbatim}	

&

\begin{Verbatim}[fontsize=\tiny]
class Circle {
  private final double pi = 3.14159;
  private double radius;
  
  public Circle(double r) {
    radius = r;
  }
  
  public double area() {
    return pi * radius * radius;
  }
  
  public double circum() {
    return 2 * pi * radius;
  }
}
public class Main {
  public static void main(String[] args){
    Circle circle = new Circle(1.6);
    double area = circle.area();
    double circumference = circle.circum();
  }
}
\end{Verbatim}	

\end{tabular}

\end{frame}

\end{document}